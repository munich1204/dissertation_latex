\begin{document}

\frontmatter

\title{Methodologies for Environmental flows}
\author{Ryan R Morrison}
\degreesubject{Ph.D., Civil Engineering}
\degree{Philosophy of Science \\ Civil Engineering}
\documenttype{Dissertation}
\previousdegrees{B.S., Civil Engineering, Washington State University, 2003
\\M.S., Civil Engineering, Washington State University, 2006}
\date{March, \thisyear} % leave blank for today's date
\maketitle
\makecopyright

\begin{dedication}
   To my parents, Albert II and Gladys, for their support,
   encouragement and the Corvette they're giving me for graduation. \\[3ex]
   ``A bird in hand is worth two in the bush''
         -- Anonymous
\end{dedication}

\begin{acknowledgments}
   \vspace{1.1in}
   I would like to thank my advisor, Professor Martin Sheen, for his support
   and some great action movies.  I would also like to thank my dog, Spot,
   who only ate my homework two or three times.  I have several other people
   I would like to thank, as well.\footnote{To my brother and sister, who
   are really cool.}
\end{acknowledgments}

\maketitleabstract %(required even though there's no abstract title anymore)

\begin{abstract}
   The theory of relativity is a real ``toughie'' to prove, but with the
   help of my family and my great grandpa Al, this paper presents the
   proof in its entirety.  Most of the math is correct, and the
   part about ``warp speed'' and ``parallel universe'' sounds very high-tech.
\clearpage %(required for 1-page abstract)
\end{abstract}

\tableofcontents
\listoffigures
\listoftables

\begin{glossary}{Longest  string}
   \item[$a_{lm}$]
      Taylor series coefficients, where $l,m = \{0..2\}$
   \item[$A_{\bf{p}}$]
      Complex-valued scalar denoting the amplitude and phase.
   \item[$A^T$]
      Transpose of some relativity matrix.
\end{glossary}

\mainmatter


\setlength{\parskip}{0.2cm} % Set the spacing between paragraphs
\setlength{\parindent}{1cm} % Remove all paragraph indentation
\setlength{\overfullrule}{0pt} % Prevent black boxes from appearing at end of lines

\linespread{1.3} % A line spread factor of 1.3 produces 1.5 lines spacing (not very intuitive, I know)

\raggedbottom % Allow page bottoms to be empty so the paragraphs aren't stretched across the page at the end of sections