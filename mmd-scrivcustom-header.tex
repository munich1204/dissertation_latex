\documentclass[12pt,oneside]{memoir}
\usepackage{layouts}[2001/04/29]
% \usepackage{manuscript}

% Use default packages for memoir setup
%
%        Default packages for memoir documents created by MultiMarkdown
%

\usepackage{fancyvrb}                        % Allow \verbatim et al. in footnotes
\usepackage{graphicx}                        % To enable including graphics in pdf's
\usepackage{booktabs}                        % Better tables
\usepackage{tabulary}                        % Support longer table cells
\usepackage[utf8]{inputenc}                % For UTF-8 support
\usepackage[T1]{fontenc}                % Use T1 font encoding for accented characters
\usepackage{xcolor}                                % Allow for color (annotations)
\usepackage{listings}                        % Allow for source code highlighting
\usepackage[sort&compress]{natbib} % Better bibliography support
\usepackage{glossaries}


% Configure default metadata to avoid errors
%
%        Configure default metadata in case it's missing to avoid errors
%

\def\myauthor{Author}
\def\defaultemail{}
\def\defaultposition{}
\def\defaultdepartment{}
\def\defaultaddress{}
\def\defaultphone{}
\def\defaultfax{}
\def\defaultweb{}


\def\mytitle{Title}
\def\subtitle{}
\def\keywords{}


\def\bibliostyle{plain}
% \def\bibliocommand{}

\def\myrecipient{}

% Overwrite with your own if desired
%\input{ftp-metadata}
